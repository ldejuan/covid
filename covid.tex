\documentclass{article}


\usepackage{arxiv}

\usepackage[utf8]{inputenc} % allow utf-8 input
\usepackage[T1]{fontenc}    % use 8-bit T1 fonts
\usepackage{hyperref}       % hyperlinks
\usepackage{url}            % simple URL typesetting
\usepackage{booktabs}       % professional-quality tables
\usepackage{amsfonts}       % blackboard math symbols
\usepackage{nicefrac}       % compact symbols for 1/2, etc.
\usepackage{microtype}      % microtypography
\usepackage{lipsum}

\title{Analysis of Covid Desease rate with DataScience }


\author{
  Gregory Sidier \thanks{Free Author} \\
  London\\
  \texttt{greg.sidier@gmail.com} \\
  %% examples of more authors
   \And
Luis de Juan \thanks{Free Author} \\
  Toulouse\\
  \texttt{luis.dejuan@gmail.com} \\
}

\begin{document}
\maketitle

\begin{abstract}
Using data science ans scientific understanding of epidemiologic rate show that the main drivers for COVID-19 contations are : to be defined. 
\end{abstract}

% keywords can be removed
\keywords{Survival Models \and Intensity Rates \and Virus}

\section{Introduction}

\section{Continuous Case}
\label{sec:continuous_case}
Let be $N(t)$ the number of fatalities at a given $t$ from a reference date $t_0$. Then :
\begin{equation}
N(t) = N_0(x)e^{\lambda_0 + \lambda_1(t,x) t}
\end{equation}
where $x$ are the features, and $\lambda_0 + \lambda_1(t,x) t$ is the fatality rate. By taking the $\ln$, we have a linear equation in $\ln(N_0)$
\begin{equation}
ln(N) = (ln(N_0) + \lambda_0)+ \lambda_1(t,x) t) 
\end{equation}
See Section \ref{sec:continuous_case}

\section{Discrete Case}
\label{sec:discrete_case}
This is also interesting to develop a discrete case where the time line is a list of period $\Delta T_i$. Then the change in fatality  
\begin{equation}
dN_i = N_{i+1} - N_i = N_i f_i
\end{equation}
is approximate by the instantaneous fatality rate $f_i$ at period $i$
\begin{equation}
N_n = N_0\prod_{i=0}^{i=n}(1+f_i)
\end{equation}
taking $\ln$
\begin{equation}
ln(N_n) = ln(N_0)+ \sum_{i=0}^{i=n}(1+f_i)
\end{equation}
\section{Features}

\subsection{Figures}
\lipsum[10] 
See Figure \ref{fig:fig1}. Here is how you add footnotes. \footnote{Sample of the first footnote.}
\lipsum[11] 

\begin{figure}
  \centering
  \fbox{\rule[-.5cm]{4cm}{4cm} \rule[-.5cm]{4cm}{0cm}}
  \caption{Sample figure caption.}
  \label{fig:fig1}
\end{figure}

\subsection{Tables}
\lipsum[12]
See awesome Table~\ref{tab:table}.

\begin{table}
 \caption{Sample table title}
  \centering
  \begin{tabular}{lll}
    \toprule
    \multicolumn{2}{c}{Part}                   \\
    \cmidrule(r){1-2}
    Name     & Description     & Size ($\mu$m) \\
    \midrule
    Dendrite & Input terminal  & $\sim$100     \\
    Axon     & Output terminal & $\sim$10      \\
    Soma     & Cell body       & up to $10^6$  \\
    \bottomrule
  \end{tabular}
  \label{tab:table}
\end{table}

\subsection{Lists}
\begin{itemize}
\item Lorem ipsum dolor sit amet
\item consectetur adipiscing elit. 
\item Aliquam dignissim blandit est, in dictum tortor gravida eget. In ac rutrum magna.
\end{itemize}


\bibliographystyle{unsrt}  
%\bibliography{references}  %%% Remove comment to use the external .bib file (using bibtex).
%%% and comment out the ``thebibliography'' section.


%%% Comment out this section when you \bibliography{references} is enabled.
\begin{thebibliography}{1}

\bibitem{kour2014real}
George Kour and Raid Saabne.
\newblock Real-time segmentation of on-line handwritten arabic script.
\newblock In {\em Frontiers in Handwriting Recognition (ICFHR), 2014 14th
  International Conference on}, pages 417--422. IEEE, 2014.

\bibitem{kour2014fast}
George Kour and Raid Saabne.
\newblock Fast classification of handwritten on-line arabic characters.
\newblock In {\em Soft Computing and Pattern Recognition (SoCPaR), 2014 6th
  International Conference of}, pages 312--318. IEEE, 2014.

\bibitem{hadash2018estimate}
Guy Hadash, Einat Kermany, Boaz Carmeli, Ofer Lavi, George Kour, and Alon
  Jacovi.
\newblock Estimate and replace: A novel approach to integrating deep neural
  networks with existing applications.
\newblock {\em arXiv preprint arXiv:1804.09028}, 2018.

\end{thebibliography}


\end{document}
